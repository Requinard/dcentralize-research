\chapter{Wat zit er in een standaard SDK}

Een groot gedeelte van de bruikbaarheid van een SDK is afhankelijk van hoe makkelijk het is om hiermee te ontwikkelen. Dit betekent dat er genoeg tutorials beschikbaar moeten zijn, dat vragen snel beantwoord worden en dat er met de community gepraat word.

\section{Tutorials}

Het voorbereiden van tutorials is essentie\"{e}l voor het bevorderen van het gebruik van een library of een SDK. Tutorials zorgen voor een gemakkelijk en geleid instappunt voor het gebruiken van de SDK, door middel van een klein projectje dat laat zien wat de SDK wel niet te bieden heeft. \\

Tutorials zijn het effectiefst als zij een vooropgezet project incrememnteel updaten. Hierbij is het een slim idee om een van de basiscomponenten te behandelen. 

\section{Voorbeelden}

Voorbeelden zijn gemakkelijk in gebruik te nemen door gebruikers die al enkele kennis hebben van het platform. Deze developers kunnen dan zonder tutorial aan de slag gaan. \\

Daarnaast zijn voorbeelden goed om aan te geven welke manieren om iets te doen de voorkeur hebben. In feite wordt er een best-practice cursus gegeven over het platform.


\section{Documentatie}

Bovenstaande is op zich al een documentatie. Voor iedere beschikbare functie is echter gedetailleerde documentatie nodig. Dit heeft vooral te amekn met return-types, parameter informatie en exceptions die voor kunnen komen. \\

Als deze samen in een grote, doorzoekbare wiki zitten word navigatie gemakkelijk gemaakt en hoeft er minder support geleverd te worden.


