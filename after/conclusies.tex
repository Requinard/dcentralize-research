\chapter{Conclusies}

Uiteindelijk zal er toch functionaliteit geimplementeerd moeten worden. Aan de hand van dit onderzoek kunnen we de te implementeren onderdelen in 2 categorieen onderverdelen.

\section{Implementatie}

Om te zorgen dat developers nette applicaties kunnen ontwikkelen zal de api bepaalde functies uit moeten kunnen voeren, waaronder het opslaan van data, dorogeven van requests en interfacing aan te bieden met appsemble zelf. \\

Hierbij is het het belangrijkste dat de api data op kan slaan en deze later op kan halen. Dit zorgt ervoor dat de basis van de implementatie er ligt. \\

Daarna is het belangrijk dat er een interface bestaat tussen appsemble en de extensions. Dit betekent dat appsemble hooks moet kunnen versturen die een extension kan interpeteren. \\

Het is ook belangrijk dat de extension bepaalde metadata over de applicatie kan opvragen. Hieronder vallen simpele dingen, zoals de naam van de app, welk kleurenschema gebruikt word en wat voor versie het is. \\

De rest van de functionaliteit is fijn om te hebben, maar is niet essentieel voor het maken van een functionerende sdk binnen appsemble. 

\section{Documentatie}

De andere helft van een sdk is de documentatie die nodig is om deze te begrijpen. Dit kan bestaan uit verschillende onderdelen waar een gebruiker mee aan de slag kan. \\

Hieronder vallen vooral voorbeelden. Zij geven aan hoe een functie aangeroepen moet worden en wat ervan verwacht kan worden. \\

Hierbovenop is het handig om documentatie te hebben voor de individuele onderdelen van de code. Dit betekent dat een functie zichzelf moet documenteren. De beste manier om dit aan te pakken is via javadocs en zelf-beschrijvende functies.