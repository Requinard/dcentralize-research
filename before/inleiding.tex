\chapter{Inleiding}

Appsemble als platform is een snel-groeiend platform voor het ontwikkelen van smartphones applicaties zonder enige kennis te hebben van programmeren. Via een simpele webapplicatie kunnen eindgebruikers een applicatie in elkaar knutselen, testen en doorsturen naar de relevante app-stores (Namelijk de Apple App Store en de Google Play Store). \\

Echter zijn componenten op het moment zeer gelimiteerd. Zij moeten namelijk ontwikkeld worden door het Appsemble team. Personen en bedrijven kunnen suggesties doen voor gewenste componenten, waarna d-centralize hiernaar gaat kijken. \\

Dit process is echter gecompliceerd en het kan een tijd duren voordat deze onderdelen ge\"{i}mplementeerd zijn. Het bedrijf moet in dit geval dus wachten tot de aanvraag in behandeling word genomen. \\

Dit heeft geleid tot de volgende vraag; Kunnen wij niet zelf componenten ontwikkelen? \\

Op het moment is dit onmogelijk. Als eerste kunnen externe developers niet bij de bron van Appsemble. Dit betekent dat zij geen componenten kunnen maken en testen. Daarnaast is er geen enkele API beschikbaar om in Appsemble zelf externe componenten te faciliteren. Hier moet verandering in komen.\\

De beginstap van het ontwikkelen hiervan is dit onderzoek, waarin geprobeerd word een antwoord te geven op de volgende vraag: \emph{Hoe wordt het process voor het ontwikkelen van plug-in componenten voor Appsemble zo simpel mogelijk gemaakt?}